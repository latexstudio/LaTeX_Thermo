%TODO的使用
%%TODO 有关内容
%%FIXME 格式的bug
%%HACK 粗鄙技巧

%TODO——注意事项
%% 脚注段落缩进,带框公式不加标点
%% 理想气体、可逆、绝热等公式特殊标注
%% 绝热:Adiabatic 等压:Isobaric 等容:Isochoric 等温:Isothermal 等熵:Isentropic 等焓:Isenthalpic 多方:Polytropic 准静态:Quasistatic 

\documentclass[UTF8]{ctexbook}

%TODO——宏包
%页面格式
\usepackage{geometry}
	\geometry{a4paper, twoside, left = 2.54 cm, right = 2.54 cm, top = 3.18 cm, bottom = 3.18 cm, headheight = 3 cm}
\usepackage{fancyhdr}
	\pagestyle{fancy}
	\fancyhf{}
	\fancyhead[EL]{\nouppercase{\CJKfamily{楷体} \leftmark}}
	\fancyhead[OR]{\nouppercase{\CJKfamily{楷体} \rightmark}}
	\fancyfoot[C]{\thepage}
	\renewcommand{\headrulewidth}{0 pt}

\usepackage{titlesec}
%	\newcommand{\sectionbreak}{\clearpage}

%交叉引用、超链接、PDF书签
\usepackage[hyperfootnotes=false]{hyperref}
	\hypersetup{
		bookmarksopen = true,
		bookmarksopenlevel = 1,
		pdfborder = 0 0 0
	}%FIXME:PDF书签有公式会报错,Latex入门 p173

%脚注
\usepackage[stable, perpage]{footmisc}
\usepackage{pifont}
	\renewcommand{\thefootnote}{\ding{\numexpr171+\value{footnote}}}
%\interfootnotelinepenalty = 10000

%数学支持
\usepackage{amsmath}
	\numberwithin{equation}{section}
	\renewcommand{\theequation}{\thesection.\arabic{equation}} %公式编号样式x.x.x
\usepackage{mathtools}
\usepackage[adobe-garamond]{mathdesign}
%\usepackage[urw-garamond]{mathdesign} %免费字体
\usepackage[ntheorem]{empheq}
\usepackage{bm}
%\usepackage{pifont} %已在“脚注”中调用
	\newcommand{\cmark}{\ding{51}}
	\newcommand{\xmark}{\ding{55}}

%中文支持
\usepackage{xeCJK}
	\xeCJKsetup{
%		PunctStyle = quanjiao,
		AutoFakeBold = false,
		AutoFakeSlant = false
	}
	\setCJKmainfont[BoldFont = 华文中宋, ItalicFont = 华文楷体, Mapping = fullwidth-stop]{华文宋体}
	\setCJKfamilyfont{中宋}[Mapping = fullwidth-stop]{华文中宋}
	\setCJKfamilyfont{楷体}[Mapping = fullwidth-stop]{华文楷体}
	\setCJKfamilyfont{仿宋}[Mapping = fullwidth-stop]{华文仿宋}
	\setCJKfamilyfont{黑体}[Mapping = fullwidth-stop]{华文黑体}
	\newcommand{\myHeavy}{\fontspec{MyriadPro-Regular} \CJKfamily{黑体}}
%\usepackage{ctex} %已经使用了ctexbook文档类
	\setcounter{secnumdepth}{4}
	\ctexset{
		part = {
			format = {\bfseries \Huge \centering},
			name = {第, 篇}
		},
		chapter = {
			format = {\bfseries \LARGE \raggedright},
		},
		section = {
			format = {\bfseries \Large \centering},
		},
		subsection = {
			format = {\bfseries},
			name = {,、\hspace{-1 em}},
			numbering = true,
			number = \chinese{subsection},
		}
	}

%符号支持
%\usepackage{gensymb}

%单位
\usepackage{siunitx}
	\sisetup{
		number-math-rm = \ensuremath,
		inter-unit-product = \ensuremath{{}\cdot{}},
		group-digits = integer,
		group-minimum-digits = 4,
		group-separator = \text{~}
	}

%枚举
\usepackage{enumitem}
	\SetLabelAlign{leftalignwithindent}{\hspace{2.1 em} \makebox[1.6 em][l]{#1}}

%定理类环境
\usepackage[thmmarks, amsmath]{ntheorem}
	\theoremstyle{empty} %不带编号
	\theoremsymbol{\ensuremath{\Box}}
		\newtheorem{_myProof}{}
	
	\theoremstyle{empty} %不带编号
	\theoremheaderfont{\myHeavy}
	\theorembodyfont{\CJKfamily{楷体}}
	\theoremsymbol{}
%	\theoremindent \parindent
		\newtheorem{_myThm}{}
	
	\theoremstyle{plain} %带编号
	\theoremheaderfont{\myHeavy}
	\theorembodyfont{\normalfont}
	\theoremsymbol{}
%	\theoremsymbol{\ensuremath{\triangleleft}}
		\newtheorem{myExample}{例}[chapter]

%表格
\usepackage{array}
	\newcolumntype{M}{>{$}c<{$}} %数学模式,居中
\usepackage{tabularx}
	\newcolumntype{Y}{>{\centering\arraybackslash}X} %定宽居中
\usepackage{booktabs} %三线表

%图片
\usepackage{graphicx}
\usepackage{asymptote}
	\begin{asydef}
		usepackage("amsmath");
		usepackage("mathtools");
		usepackage("mathdesign", "adobe-garamond");
		usepackage("bm");
		usepackage("xeCJK");
			texpreamble("\setCJKmainfont[BoldFont = 华文中宋, ItalicFont = 华文楷体]{华文宋体}");
			texpreamble("\setmainfont{AGaramondPro-Regular.otf}");
		usepackage("siunitx");
			texpreamble("\sisetup{
				number-math-rm = \ensuremath,
				inter-unit-product = \ensuremath{{}\cdot{}},
				group-digits = integer,
				group-minimum-digits = 4,
				group-separator = \text{~}
			}");
		texpreamble("\input{Symbols.tex}");
		texpreamble("\newcommand{\emphA}[1]{{\myHeavy #1}}");
		texpreamble("\newcommand{\emphB}[1]{{\itshape #1}}");
		
		size(6 cm);
		
		defaultpen(fontsize(9 pt));
		
		pen color1 = rgb(0.368417, 0.506779, 0.709798);
		pen color2 = rgb(0.880722, 0.611041, 0.142051);
		pen color3 = rgb(0.560181, 0.691569, 0.194885);
		pen color4 = rgb(0.922526, 0.385626, 0.209179);
		pen color5 = rgb(0.647624, 0.378160, 0.614037);
		pen color6 = rgb(0.772079, 0.431554, 0.102387);
		pen color7 = rgb(0.363898, 0.618501, 0.782349);
		pen color8 = rgb(0.972829, 0.621644, 0.073362);
	\end{asydef}

%浮动体、图标标题
\renewcommand{\thefigure}{\arabic{chapter}.\arabic{figure}}
\usepackage{floatrow}
	\floatsetup[table]{capposition = top}
\usepackage{caption}
	\DeclareCaptionFont{myCationFont}{\CJKfamily{楷体} \small}
	\captionsetup[figure]{
		font = myCationFont,
		labelsep = quad,
		skip = 10 pt,
		position = bottom
	}
	\captionsetup[table]{
		font = myCationFont,
		labelsep = quad,
		skip = 10 pt,
		position = top
	}


%TODO——自定义环境
%自定义定理(定律)
\newenvironment{myThm}[1]
	{\begin{_myThm}[\hskip 2 em #1]}
	{\end{_myThm}}
\newenvironment{myThm*}
	{\begin{_myThm} \hskip 2 em}
	{\end{_myThm}}
%自定义证明(说明)
\newenvironment{myProof}
	{\begin{_myProof} \normalfont \small \CJKfamily{仿宋} \hskip 2 em}
	{\end{_myProof}}
%自定义列表
\newenvironment{myEnum1}
	{\begin{enumerate} [label = \alph*), align = leftalignwithindent, leftmargin = 0 pt, itemsep = 10 pt, parsep = 0 pt, listparindent = 2 em]}
	{\end{enumerate}}
\newenvironment{myEnum2}
	{\begin{enumerate} [label = (\Roman*), align = leftalignwithindent, leftmargin = 2 em, topsep = 0pt, itemsep = 0 pt, parsep = 0 pt, listparindent = 2 em]}
	{\end{enumerate}}
%子公式
\newenvironment{mySubEq}
	{\subequations \renewcommand{\theequation}{\theparentequation-\alph{equation}}}
	{\endsubequations
	\ignorespacesafterend}
%带框公式
\newenvironment{boxedEq}
	{\empheq[box = \fbox]{equation}}
	{\endempheq}
%大括号公式
\newenvironment{braceEq}[1][align]
	{\mySubEq
		\setkeys{EmphEqEnv}{#1}
		\setkeys{EmphEqOpt}{left = \empheqlbrace}
		\EmphEqMainEnv}
	{\endEmphEqMainEnv \endmySubEq}
%自定义表格 {format}{caption}{label}
\newenvironment{myTable}[4][htb]
	{\begin{table}[#1] \centering \small \caption{#3} \label{#4} \begin{tabular}{#2}}
	{\end{tabular} \end{table}}

%TODO——自定义命令
\input{Symbols.tex}

%强调
\newcommand{\emphA}[1]{{\myHeavy #1}}
\newcommand{\emphB}[1]{{\itshape #1}}

%引入文件
\let \oldInclude = \include
\renewcommand{\include}[1]{{\let \clearpage = \relax \oldInclude{#1}}}
%空行
\newcommand{\blankline}{\mbox{}}
%自定义列表项目(不可删除换行!)
\newcommand{\myItem}[1]{
	\item
	{\bfseries #1}
	
	
}
%引用格式
\newcommand{\secref}[1]{\secSymbol \ref{#1} }
\newcommand{\subsecref}[1]{第\ref{#1}小节}
\newcommand{\egref}[1]{例 \ref{#1}}
%公式标注
\newcommand{\myTag}[1]{\tag*{\CJKfamily{楷体} [#1]}}
\newcommand{\myTagNumbering}[1]{\refstepcounter{equation} \tag*{\CJKfamily{楷体} [#1] \, (\theequation)} }

\title{
	\vspace{-4 cm}
	\bfseries
	热力学与统计物理I
}
\author{
	\CJKfamily{楷体}
	复旦大学\phantom{空格}陈焱
}
\date{
	\CJKfamily{楷体}
	\today
}

\begin{document}
%	\maketitle
%	
%	\frontmatter
%	{\let \cleardoublepage = \clearpage
%		\chapter{绪论}
%			\include{Chapters/Front_Introduction}
%			
%		\chapter{符号表}
%			\include{Chapters/Front_Symbols}
%			
%		\chapter{缩略词表}
%			\include{Chapters/Front_Abbreviations}
%	}
%	\mainmatter
%	\part{热力学}
		\chapter{热力学基础}
			\include{Chapters/Chapter1}
			
		\chapter{均匀系统的平衡特性}
			\include{Chapters/Chapter2}
			
		\chapter{相变的热力学理论}
			我们所研究的系统是逐渐复杂的。首先是\emphB{独立子体系},如理想气体、Fermi气体、Bose气体等;之后是\emphB{近独立子体系},如准粒子气体;%TODO:20160413 翻译
本章主要探讨\emphB{相互作用体系},它包含多种物态,也会有相变的发生。

\section{热动平衡判据;开系热力学}
	在 \secref{SEC_熵增加原理与最大功}中我们已经证明,对于孤立体系(即绝热过程),熵将不断增加,直至达到极大(平衡)。因此,熵达到极大便可以作为体系达到平衡的判据。
	
	\subsection{平衡判据}
		\begin{myEnum1}
			\myItem{熵判据}
				由上,在 $\vd U = 0, \, \vd V = 0, \, \vd N = 0$ 的前提下,$\vd S = 0, \, \vd^{\:2} \! S <0$ 即说明(孤立)体系达到了平衡。这里的“$\vd\,$”表示\emphB{虚变化},与“$\dd\,$”代表的\emphB{真实变化}有所不同,它与虚功原理中的虚位移是类似的。%HACK:20160420 δS的缩进
				
				平衡分为三种:稳定平衡、亚稳平衡和不稳定平衡。如前所述,熵的极大值对应平衡态。稳定平衡对应其中\emphB{最大}的极大值,而亚稳平衡对应其他较小的极大值,即对于无限小的变动是稳定的,而对于有限的变动则是不稳定的。不稳定平衡对应极小值,虽然有 $\vd S = 0$,但却不满足 $\vd^{\:2} \! S <0$。平衡的稳定性可以用力学类比来理解。重力势能 $E_\text{p}$ 的极小值对应平衡。其最小值对应稳定平衡;但相对极小对于大的扰动是不稳定的,所以对应亚稳平衡。而极大值则对应不稳定平衡,稍有扰动就会偏离。%TODO:20160420 图片
				
				对于 $\vd S = 0$、$\vd^{\:2} \! S =0$ 的临界态,将 $S$ 围绕极值点Taylor作展开,可得
				\begin{equation}
					\tl{\incr\,} S = \vd S + \frac{1}{2!} \vd^{\:2} \! S + \frac{1}{3!} \vd^{\:3} \! S + \frac{1}{4!} \vd^{\:4} \! S + \cdots < 0 \fullstop
				\end{equation}
				式中的“$\tl{\incr\,}$”同样表示虚变动。由于要满足 $S \rightarrow -S$ 的对称性,因此 $\vd^{\:3} \! S = 0$。从而%TODO:20160420 三阶项为何等于零
				\begin{equation}
					\vd^{\:4} \! S < 0 \comma
				\end{equation}
				这就是临界态的平衡判据。
				
			\myItem{自由能判据}
				考虑一个系统与热库(即环境)组成的复合体系,它是一个孤立系。显然,总内能
				\begin{equation}
					U_0 = U_\text{sys} + U_\text{res} = \const \comma
				\end{equation}
				总体积
				\begin{equation}
					V_0 = V_\text{sys} + V_\text{res} = \const
				\end{equation}
				设想体系发生了一个虚变动,则
				\begin{braceEq}[gather]
					\vd U_\text{sys} + \vd U_\text{res} = 0 \comma \\
					\vd V_\text{sys} + \vd V_\text{res} = 0 \fullstop \label{EQ_DELTA_V_SYS+DELTA_V_RES=0}
				\end{braceEq}
				根据熵判据,
				\begin{braceEq}
					\vd S_0 &= \vd \; (S_\text{sys} + S_\text{res}) = 0 \comma \\
					\vd^{\:2} \! S_0 &= \vd^{\:2} (S_\text{sys} + S_\text{res}) < 0 \fullstop
				\end{braceEq}
				根据热力学基本微分方程 \eqref{EQ_FUNDAMENTAL_EQUATION_FOR_PVT_SYSTEM} 式,可得
				\begin{equation}
					\vd U_\text{res} = T_\text{res} \vd S_\text{res} + p_\text{res}  \vd V_\text{res} \fullstop \footnote{
						已经把式~\eqref{EQ_FUNDAMENTAL_EQUATION_FOR_PVT_SYSTEM} 中的微分“$\dd\,$”改成了变分“$\vd\,$”的形式。
					}
				\end{equation}
				
				在系统温度、体积均不变(即 $\vd T_\text{sys} = 0$、$\vd V_\text{sys} = 0$)的情形下,根据式~\eqref{EQ_DELTA_V_SYS+DELTA_V_RES=0} 可知 $\vd V_\text{res} = 0$。由于是平衡态,又有 $T_\text{res} = T_\text{sys}$。因此
				
				\begin{align}
					\vd F_\text{sys} &= \vd \; (U_\text{sys} - T_\text{sys} S_\text{sys}) \\
					&= \vd U_\text{sys} - T_\text{sys} \vd S_\text{sys} \\
					&= -\vd U_\text{res} + T_\text{res} \vd S_\text{res} \\
					&= -p_\text{res}  \vd V_\text{res} \\
					&= -p_\text{res}  \vd V_\text{sys} = 0 \comma \\
					\vd^{\:2} \! F_\text{sys} &= %TODO:20160420 推导过程
				\end{align}
				
			\myItem{Gibbs函数判据}
			\myItem{内能判据}
		\end{myEnum1}
		
	\subsection{开系热力学}
		所谓“开系”,即\emphA{开放系(open system)},它指粒子数可变且有能量交换的系统。
		
		对于封闭系,我们已经知道
		\begin{equation}
			\dd G = -S \dd T + V \dd p \fullstop
		\end{equation}
		而对于开放系,需引进\emphA{化学势} $\m$,其定义为
		\begin{equation}
			\m \eqdef \myPartial{G}{n}{T, \, p} \fullstop \footnote{这里的$n$指物质的量,而非粒子数。定义$\m \eqdef (\pd G / \pd N)_{T,\,p}$ 当然也可以,不过我们只采用第一种定义。}
		\end{equation}
		因此,
		\begin{equation}
			\dd G = -S \dd T + V \dd p + \m \dd n \fullstop
		\end{equation}
		利用Legendre变换,可得粒子数可变系统的热力学基本方程:
		\begin{braceEq}
			\dd U &= T \dd S - p \dd V + \m \dd n \comma \\
			\dd H &= T \dd S + V \dd p + \m \dd n \comma \\
			\dd F &= -S \dd T - p \dd V + \m \dd n \fullstop
		\end{braceEq}
		于是可以得出 $\m$ 的几个等价定义:
		\begin{equation}
			\m = \myPartial{U}{n}{S, \, V} = \myPartial{H}{n}{S, \, p} = \myPartial{F}{n}{T, \, V} = \myPartial{G}{n}{T, \, p} \fullstop
		\end{equation}
		
		Gibbs函数 $G(T,\,p,\,n)$ 是广延量。因此可定义\emphA{摩尔Gibbs函数} $G_\text{m}$,使得
		\begin{equation}
			G(T,\,p,\,n) = n G_\text{m} (T,\,p) \fullstop
		\end{equation}
		因此,
		\begin{equation}
			\m \eqdef \myPartial{G}{n}{T, \, p} = \left[ \frac{\pd}{\pd n} (n G_\text{m}) \right]_{T, \, p} = G_\text{m} \fullstop
		\end{equation}
		对于 \SI{1}{\mol} 物质,有
		\begin{equation}
			\dd \m = \dd G_\text{m} = -S_\text{m} \dd T + V_\text{m} \dd p \fullstop
		\end{equation}
		
		\blankline
		
		定义\emphA{巨势(grand potential)}或\emphA{巨热力学势}
		\begin{equation} \label{EQ_DEF_OF_GRAND_POTENTIAL}
			\Y \eqdef F -\m n = F - G = -p \dd V \comma
		\end{equation}
		则其微分
		\begin{align}
			\dd \Y &= \dd F - \dd \; (\m n) \notag \\
			&= -S \dd T - p \dd V + \m \dd n - (\m \dd n + n \dd \mu) \notag \\
			&= -S \dd T - p \dd V - n \dd \mu \fullstop \footnotemark
		\end{align} \footnotetext{根据式~\eqref{EQ_DEF_OF_GRAND_POTENTIAL},可知 $\dd \Y = -\dd \; (p V)$。实际上,
		\begin{align*}
			-S \dd T - p \dd V - n \dd \mu &= -S \dd T - p \dd V - n (-S_\text{m} \dd T + V_\text{m} \dd p) \\
			&= -S \dd T - p \dd V + S \dd T - V \dd p = -\dd \; (p V) \comma
		\end{align*}
		这是不矛盾的。}
		在统计物理中,巨势与巨配分函数有关。
		
\section{平衡条件与稳定条件}
	\subsection{平衡条件}
		对于一个\emphB{单元两相}(即一种组分,两种状态)的系统,其平衡条件为
		\begin{equation}
			\vd S = \vd S_1 + \vd S_2 = 0 \comma
		\end{equation}
		其中的下标“1”和“2”分别表示两个相。根据热力学基本方程,有
		\begin{braceEq}
			\vd S_1 &= \frac{1}{T_1} (\vd U_1 + p_1 \vd V_1 - \m_1 \vd n_1) \comma \\
			\vd S_2 &= \frac{1}{T_2} (\vd U_2 + p_2 \vd V_2 - \m_2 \vd n_2) \fullstop
		\end{braceEq}
		设整个系统是孤立的,则
		\begin{braceEq}
			\vd U_1 + \vd U_2 &= 0 \comma \\
			\vd V_1 + \vd V_2 &= 0 \comma \\
			\vd n_1 + \vd n_2 &= 0 \fullstop
		\end{braceEq}
		因此
		\begin{align}
			\vd S &= \vd S_1 + \vd S_2 \notag \\
			&= \left[ \frac{1}{T_1} (\vd U_1 + p_1 \vd V_1 - \m_1 \vd n_1) \right] + \left[ \frac{1}{T_2} (\vd U_2 + p_2 \vd V_2 - \m_2 \vd n_2) \right] \notag \\
			&= \left[ \frac{1}{T_1} (\vd U_1 + p_1 \vd V_1 - \m_1 \vd n_1) \right] + \left[ \frac{1}{T_2} (-\vd U_2 - p_2 \vd V_2 + \m_2 \vd n_2) \right] \notag \\
			&= \vd U_1 \left( \frac{1}{T_1} - \frac{1}{T_2} \right) + \vd V_1 \left( \frac{p_1}{T_1} - \frac{p_2}{T_2} \right) - \vd n_1 \left( \frac{\m_1}{T_1} - \frac{\m_2}{T_2} \right) = 0 \fullstop
		\end{align}
		由于 $U_1$、$V_1$ 和 $n_1$ 是相互独立的,所以它们的系数都应该等于零,即
		\begin{braceEq}
			T_1 &= T_2 \comma \quad \text{(热平衡条件)} \\
			p_1 &= p_2 \comma \quad \text{(力学平衡条件)} \\
			\m_1 &= \m_2 \fullstop \quad \text{(化学平衡条件)}%TODO:20160420 相变平衡条件
		\end{braceEq}
		若 $T$ 不等,则仍有能量流动;$p$ 不等,则两相界面可以移动;$\m$ 不等,则每一相中的粒子数仍在变化,因此都不是平衡。
		
	\subsection{稳定条件}
		计算熵的二级变分:
		\begin{align}
			\vd^{\:2} \! S_1 = \frac{1}{2} \left( \frac{\pd^{\:2} \! S_1}{\pd U_1^2} \right) %TODO:20160420 (δS)^2还是δ^2 S
		\end{align}
		
\section{单元系的复相平衡;相变分类;Ehrenfest公式}
\section{气液相变;临界点行为}
\section{临界指数;普适性}
\section{Landau平均场理论}
	
\raggedbottom%FIXME:20260325 交叉引用、脚注每页重新计数失效,必须加上该行
\pagebreak
			
%		\chapter{多元复相 热力学第三定律}
%			\include{Chapters/Chapter4}
%			
%	\part{统计物理}
%		\chapter{统计物理学基本概念}
%			\section{微观态的经典及量子描述}
%\section{}
%\section{}
%\section{}

		
%		\chapter{统计}
%			\include{Chapters/Chapter6}
%		
%		\chapter{系综理论}
%			\include{Chapters/Chapter7}
%		
%		\chapter{相变的统计物理简介}
%			\include{Chapters/Chapter8}
		
\end{document}